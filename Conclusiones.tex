\chapter{Conclusiones}
\label{chap:Conclusiones}
\Abstract{El presente capítulo expone las conclusiones finales del proyecto extraídas a partir de los
resultados obtenidos de las diferentes pruebas realizadas.}

Durante la elaboración de este proyecto se ha perfeccionado un sistema compuesto por una cámara de color y profundidad, un brazo robótico y un ordenador con MATLAB. El sistema ha sido diseñado para identificar, detectar y recoger piezas de LEGO colocadas sobre la mesa de trabajo de un brazo IRB120.

El objetivo final de este proyecto es el perfeccionamiento del sistema previamente desarrollado por Ana Berjón Valles \citep{TFGAna} mejorando la capacidad de reconocimiento y la conexión con el brazo robótico. Para ello se plantearon múltiples objetivos para cumplir a lo largo del desarrollo del proyecto, estos se pueden ver en la \autoref{sec:Objetivos}. Todos los objetivos planteados durante el desarrollo del proyecto se han podido llevar a cabo y se ha mejorado en su totalidad el sistema. A continuación, se va desarrollar más en detalle los objetivos cumplidos.

Uno de los principales objetivos consiste en subsidiar los problemas del sistema antiguo ante cambios de iluminación. Como el sistema clásico se basa puramente en filtros de color y detección de borde, se ve notablemente afectado por cambios en la iluminación. Falta comprobar el funcionamiento de los sistemas desarrollados durante este proyecto en el laboratorio, pero con los resultados obtenidos durante la evaluación de estos se considera que este problema ya ha sido solventado y se ha mitigado el problema de falsos positivos o piezas no detectadas. Dados los resultados se recomienda para futuros proyectos la implantación del sistema basado en YOLO con LEGO16 y el modelo de regresión basado en LEGONet. También es recomendable plantearse el uso de un sistema combinado con varias redes neuronales.

Otra gran limitación del sistema anterior era los fallos debidos al análisis de profundidad. Debido al tipo de tecnología empleada para detectar la profundidad, esta se ve bastante afectada por los reflejos de las luces del laboratorio sobre las piezas de LEGO. Este problema ha podido ser solventado mediante la modificación del sistema de análisis de profundidad. Se ha introducido la posibilidad de calibrar rápidamente la cámara para poder tomar referencias de alturas. Además, se ha repartido el área de trabajo en secciones y se ha calculado la profundidad de dichas secciones con la mediana y teniendo en cuenta el entorno a la sección. De esta forma se mitiga el efecto de los reflejos.

Generalización de los algoritmos empleados. Durante el desarrollo del proyecto se ha tenido en cuenta este objetivo y es por ello que todo el procesado de la imagen no depende de la cámara, del brazo ni del laboratorio. La cámara ha sido tratada como una clase y es por ello que con solo modificar esta se puede usar el sistema con cualquier cámara. Además, el proceso de segmentación ya no depende tanto del calibrado de color de la cámara empleada. Para el procesado de la imagen de profundidad se ha creado un sistema fácil de calibrar y reutilizar para diferentes circunstancias. Se puede emplear para cualquier tipo de pieza y el único parámetro que se debe de cambiar es la altura de la pieza medida para determinar el número de piezas apiladas. Todo proceso que involucre al brazo robótico es llevado a cabo en forma de funciones, por ello solo modificando estas se puede implantar el sistema en cualquier otro brazo robótico del laboratorio.

Analizando el proyecto de forma global, este nuevo sistema supone un gran avance frente al anterior sistema y una vez sea implantado supondrá una gran mejora en la tecnología y capacidades de los robots industriales de Comillas ICAI.