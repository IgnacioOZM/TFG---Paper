\chapter{Futuros desarrollos}
\label{chap:Futuros desarrollos}
\Abstract{En este capítulo se desarrollan las posibles mejoras que se pueden llevar a cabo partiendo del trabajo desarrollado en este proyecto. Se plantean posibles objetivos y posibles formas para llevarlos a cabo.}

Este proyecto es la evolución de un proceso amplio y ambicioso, por lo que todavía hay numerosos puntos de mejora. El área de la robótica y la inteligencia artificial presenta infinidad de salidas, es por ello que en este apartado solo nos vamos a centrar en el sistema actual dentro de sus capacidades y limitaciones. Este capítulo se va a dividir en dos secciones, en primer lugar se va a desarrollar las posibles mejoras en el reconocimiento de imagen y en segundo lugar se hablará de mejoras en el agarre y colocación de piezas a través del brazo robótico.

Con este proyecto se ha mejorado notablemente el reconocimiento de imagen y se ha evolucionado el anterior sistema con la inclusión de nuevas tecnologías como las redes neuronales. Sin embargo, este sistema no es perfecto y puede ser ampliamente mejorado.
\begin{itemize}
\item Reentrenar las redes: debido a limitaciones por \textit{hardware}, algunas redes no han podido mostrar su potencial o no han podido ser entrenadas. Por ello, es recomendable reentrenarlas con mayor potencia y a ser posible con una mayor base de datos.
\item Piezas dobles: para mejorar la capacidad del sistema se puede plantear el uso de piezas de LEGO dobles. Para ello es necesario reentrenar las redes neuronales y obtener una nueva base de datos que incluya las piezas dobles. Además, se requiere de un sistema que sea capaz de distinguir una pieza doble de dos piezas simples del mismo color muy próximas. Para ello una posible opción es un análisis detallado de la pieza mediante la detección de borde. En caso de ser dos piezas simples debe de existir un borde en medio que las separe.
\item Reflejos: el sistema ha mejorado frente a los efectos de los reflejos gracias al nuevo análisis de profundidad. Sin embargo, todavía se cree que se puede mejorar más. Para ello se pueden plantear varias alternativas. Se puede usar un filtro de polarización que ayude a reducirlos aunque en ese caso habría que analizar el impacto que tiene sobre los sensores de la cámara. O se puede plantear la alternativa de tomar múltiples fotos desde diferentes ángulos.
\end{itemize}

Durante el desarrollo de este proyecto se ha trabajado poco en el desarrollo de las capacidades del brazo robótico y su conexión con MATLAB. Es por ello por lo que este todavía puede ser ampliamente mejorado.
\begin{itemize}
\item Automatización: el sistema actual necesita que una persona indique cuando mandar la información al robot y como. Esto hace que el sistema no pueda ser independiente y sea más lento de lo necesario. Es por ello por lo que se aconseja rediseñar el sistema de comunicación entre MATLAB y el brazo robótico para que no sea necesaria la intervención humana. Esto implica la creación de una comunicación bidireccional.
\item Sistema anticolisión: en el estado actual no se hace ningún análisis del proceso que debe de llevar a cabo el robot para coger una pieza. Esto implica que se puede dar el caso en el que este colisione con otra pieza. Por ello se recomienda la elaboración de un sistema que tenga en cuenta las limitaciones físicas del robot y la disposición de las piezas para poder maniobrar sin colisionar. Es necesario también tener en cuenta estas consideraciones a la hora de depositar las piezas ya que en caso de que haya más piezas azules que rojas es muy probable que la cámara colisione contra estas.
\item Variabilidad de las coordenadas: el sistema de coordenadas presenta imperfecciones derivadas de la variabilidad de las coordenadas respecto al punto de referencia. Esta contrariedad se podría resolver de varias maneras, una de ellas es modificar el diseño de la pieza de manera que las tuercas quedaran encajadas impidiendo así su rotación. La otra posibilidad planteada, que no es incompatible con la anterior, sería realizar un proceso iterativo, que se podría combinar con el problema de la automatización. En este caso, la señal que debería enviar el robot contendría la posición del mismo en ese preciso instante, información obtenida con la instrucción CRobT y que permitiría que MATLAB calculara en ese momento la información de la siguiente pieza. Esto reduciría la incertidumbre del conjunto y lo haría más resistente a posibles imprevistos. Para que esto funcione es necesario primero poder establecer una comunicación bidireccional con el robot.
\item Piezas rotadas apiladas: con el sistema actual se asume que si una pieza esta rotada no puede tener ninguna pieza más apilada. Para llevar esto a cabo es necesario modificar el sistema de alturas con el que trabaja el sistema actual de forma que pueda reconocer estas piezas y añadirles una altura de 3mm respecto a una pieza de la misma altura pero enganchada. Una vez reconocida la pieza se debería de recolocar en la hilera más cerca para de esta forma poder separar las piezas y recolocar las donde corresponda.
\end{itemize}

Por último, otro posible desarrollo del proyecto es la traspaso de este a otro lenguaje de programación. MATLAB es un lenguaje muy potente y con muchas ayudas que han sido empleadas para el desarrollo del proyecto. Sin embargo, no destaca por su rapidez y carece de un buen grado de control ya que no permite el control de los hilos de ejecución. Se recomienda el uso de C++ o Python ya que estos se caracterizan por ser más rápidos y dan un mayor control sobre la ejecución. Python destaca por ser uno de los lenguajes más empleados para el trabajo con redes neuronales y disponer de amplias librerías con las que poder trabajar.
